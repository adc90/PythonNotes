%------------------------------------------------------------------
% Python notes
%------------------------------------------------------------------
\documentclass[letterpaper,11pt]{article}
%---------------------
% Packages
%---------------------
\usepackage[margin=1.0in]{geometry}
\usepackage{tabularx}
\usepackage{xcolor}
\usepackage[most]{tcolorbox}

\tcbset{
    frame code={}
    center title,
    left=0pt,
    right=0pt,
    top=0pt,
    bottom=0pt,
    colback=gray!40,
    colframe=white,
    width=\dimexpr\textwidth\relax,
    enlarge left by=0mm,
    boxsep=5pt,
    arc=0pt,outer arc=0pt,
}
%---------------------

%---------------------
% Macros & Misc
%---------------------
\topmargin=-1.0in
\setlength{\textheight}{10in}
\pagenumbering{gobble}
%---------------------
% Document start
%---------------------
\begin{document}
\begin{flushleft}
    \LARGE{\textbf{Python notes}}
\end{flushleft}
\textbf{Python Intro}
\par{Python is an extremely powerful and dynamic object-oriented programming.
There are several reasons to use Python. One of the biggest draws is
portability. Python runs on almost every operating system. In addition python is
fairly easy to pick up}
\\
\textbf{Invoking Python}
\par{One \emph{windows} you invoke the python interpreter by launching the
\texttt{cmd} utility and entering in \texttt{python} which should give you a
prompt similar to the one bellow.}
\begin{tcolorbox}
\begin{footnotesize}
\begin{verbatim}
Python 2.6.6 (r266:84297, Aug 24 2010, 18:46:32) [MSC v.1500 32 bit (Intel)] on win32
Type "help", "copyright", "credits" or "license" for more information
>>>
\end{verbatim}
\end{footnotesize}
\end{tcolorbox}
\textbf{Functions}
\par{One of the most basic building blocks of any programming language is the
function. In python functions are defined with the \textem{def} keyword.
Functions have two main parts, the function declaration and the function body. }
\\
\begin{minipage}{.5\textwidth}
\begin{tcolorbox}
\begin{footnotesize}
\begin{verbatim}
>>> def functionName():          #Decleration
>>>    print "Hello World!"      #Function body
...
>>> functionName()
>>> Hello World!
\end{verbatim}
\end{footnotesize}
\end{tcolorbox}
\end{minipage}
\par{Functions can also take take arguments. These arguments can be passed in
between the parentheses of the function definition. These arguments can then be
accessed in the body of the function.}
\begin{minipage}{.5\textwidth}
\begin{tcolorbox}
\begin{footnotesize}
\begin{verbatim}
>>> def argumentFuction(a, b):
>>>    print "A=" + a + " B=" + b
...
>>> argumentfuction(1,2)
>>> A=1 B=2
\end{verbatim}
\end{footnotesize}
\end{tcolorbox}
\end{minipage}
\\
\textbf{Variables}
\par{Another basic building block of any programming language is variables.
Variables consist of three components \textem{type}, \textem{indentifier}, and
a \textem{value}. \texttt{Python} is what is known as a \texttt{dynamic}
language. What this means is that you are not required to include a type
delcration, in contrast to strongly typed languages such as \texttt{Java} or
\texttt{C++}.}
\\ \\
%
\begin{minipage}{.5\textwidth}
    \small \textbf{Java}
    \begin{tcolorbox}
        \begin{footnotesize}
            \begin{verbatim}
            >>> int number = 5;
            ...
            >>> System.out.print(number);
            ...
            >>> 5
            \end{verbatim}
        \end{footnotesize}
    \end{tcolorbox}
\end{minipage}
%
\begin{minipage}{.5\textwidth}
    \small \textbf{Python}
    \begin{tcolorbox}
        \begin{footnotesize}
            \begin{verbatim}
            >>> number = 5
            ...
            >>> print(5)
            ...
            >>> 5
            \end{verbatim}
        \end{footnotesize}
    \end{tcolorbox}
\end{minipage}
\\
\textbf{Conditional}
\par{Python supports conditional statements in the form of \textem{if, else}
and \textem{elif}. The way if statements work is the code below the if statement
is executed if the condition is true.} \\ \\
\begin{minipage}{.5\textwidth}
    % \small \textbf{If statement}
    \begin{tcolorbox}
        \begin{footnotesize}
            \begin{verbatim}
            >>> x = True
            >>> if x == True:
            >>>     print "x is true"
            >>> elif x == False:
            >>>     print "x is false"
            >>> else:
            >>>     print "x is neither"
            \end{verbatim}
        \end{footnotesize}
    \end{tcolorbox}
\end{minipage}
\\ \\
\clearpage
\textbf{Looping Constructs}
\par{Python features several constructs to repeat a statement. The first of
these explressions is the \texttt{while} loop. Thie while loop repeats the code
below the loop while the condition is true.}
\\ \\
\begin{minipage}{.5\textwidth}
    \small \textbf{While loop}
    \begin{tcolorbox}
        \begin{footnotesize}
            \begin{verbatim}
            >>> x = 1
            >>> while x < 10:
            >>>     print(str(x) + ","),
            >>>     x += 1
            ...
            >>> 1,2,3,4,5,6,7,8,9,
            \end{verbatim}
        \end{footnotesize}
    \end{tcolorbox}
\end{minipage}
\par{Another type of loop in Python is the \texttt{for} loop. The syntax of the
for loop is \texttt{for item in sequence: block}. Each loop item is set to the
next item in the sequence and the block of code is executed.}
\\ \\
\begin{minipage}{.5\textwidth}
    \small \textbf{For loop}
    \begin{tcolorbox}
        \begin{footnotesize}
            \begin{verbatim}
            >>> word = "Python"
            >>> for ch in word:
            >>>     print(ch + "-"),
            ...
            >>> P-y-t-h-o-n-
            \end{verbatim}
        \end{footnotesize}
    \end{tcolorbox}
\end{minipage}
\\
\textbf{Python Classes}
\par{Object oriented programming \textit{(O.O.P.)} is a programming paradigm that is used to
    model real world objects and situations. Two of the main concepts involved
in \textit{O.O.P.} are classes and objects. To make an analogy to the real
world classes are like a blue print for a house and objects would be instances
of that house.}
\par{Classes in Python are basically a collection of variables and functions. In
the context of object oriented programming functions are referred to as methods
and variables are referred to as fields, or attributes. Lets take a look at an
example bellow.}
\\ \\
\begin{minipage}{.5\textwidth}
    \small \textbf{Classes}
    \begin{tcolorbox}
        \begin{footnotesize}
            \begin{verbatim}
            >>> class test():
            >>>    word = "Hello World"
            >>>    def printWord(self):
            >>>       print self.word
            >>>
            >>> t = test()
            >>> t.printWord()
            >>>
            >>> Hello World
            \end{verbatim}
        \end{footnotesize}
    \end{tcolorbox}
\end{minipage}
\par{In the example above the class declaration starts with the keyword
\texttt{class} followed by an identifier and a pair of parentheses. Following
the declaration we have the variable definition \texttt{word} which is
attached to the \texttt{test} class. Similarly we have a function attached to the
test class called \texttt{printWord} which will print the contents of the
variable word.}
\end{document}
